% ==============================================================================



\documentclass[../../../Master/AppliedStochastics.tex]{subfiles}



% ==============================================================================


\author{Helen}
\date{22 October 2018}


% ==============================================================================
%
\begin{document}
%
% ==============================================================================


\makelecture


\section{Examples}

\begin{example}{Motivating example: Moments of the Poisson}
    Let $(N_t)_{t \geq 0}$ be a Poisson counting process
        where $N \sim \PPP(\text{constant rate 1})$ on $[0, \infty)$.
    The fact that $(N_t)_{t \geq 0}$ is a Poisson counting process means
        that $N_t \in \bbZ$ and $N_t$ ``jumps by 1 unit at rate 1'',
        ie. $N_t = N([0, t]) + N_0$. 
    Note that writing $N \sim \PPP(\text{constant rate 1})$
        is the same as writing $N \sim \PPP(\lambda)$,
            where $\lambda$ is the Lebesgue measure.
    
    
    We want to answer the question:
        What is $\E[N_t^k]$?
    First we define
    $$\begin{aligned}
        \E^x[f(N_t)] \defeq \E[f(N_t) | N_0 = x ]\,.
    \end{aligned}$$
    We see that
    $$\begin{aligned}\label{condexp1}
        \E^x[f(N_t)] = f(x) + \int_0^t \frac{d}{ds} \E^x [f(N_s)] ds
                    &= f(x) + \int_0^t \lim_{\epsilon \to 0}
                        \frac{1}{\epsilon} \E^x [f(N_{s+\epsilon}) - f(N_s)]
                        \dif s\\
                    &= f(x) + \int_0^t  \lim_{\epsilon \to 0}
                        \frac{1}{\epsilon} \E^x [
                            \E^s [f(N_{s+\epsilon}) - f(N_s) | N_s] ] \dif s
    \end{aligned}$$
    This motivates the following definition.
    \begin{definition}
        Let
        $$\begin{aligned}
            (Gf)(y) \defeq \lim_{\epsilon \to 0} \frac{1}{\epsilon}
                            \E^{y}[f(N_\epsilon) - f(y)]\,.
        \end{aligned}$$
    The function $G$ is called the \defn{generator} for $N_t$. 
    \end{definition}
    
    
    Since $N_\epsilon - N_0 \sim \Poisson(\epsilon)$, 
    $$\begin{aligned}
        Gf(x) &= \lim_{\epsilon \to 0} \frac{1}{\epsilon}
                    \E^{x} [f(N_\epsilon) - f(x)] \\
              &= \lim_{\epsilon \to 0} \frac{1}{\epsilon}
                    \E[f(N_\epsilon) - f(x) | N_0 = x] \\
              &= \lim_{\epsilon \to 0}  \frac{1}{\epsilon}
                    (\P \{ N_\epsilon - N_0 = 0 \} ( f(x) - f(x) ) +
                    \P \{ N_\epsilon - N_0 = 1 \} ( f(x+1) - f(x) ) \\
              &\qquad + \P \{ N_\epsilon - N_0 \geq 2 \} \E[f(N_\epsilon)
                      - f(x) | N_\epsilon - N_0 \geq 2 ] ) \\
              &= \lim_{\epsilon \to 0}  \frac{1}{\epsilon}
                    \left( e^{-\epsilon} \cdot \epsilon \cdot (f(x+1) - f(x)) + \sum_{n \geq 2} \dfrac{e^{-\epsilon} \epsilon^n}{n!}
                        \E[f(N_\epsilon) - f(x) | N_\epsilon - N_0 \geq 2 ]
                    \right) \\
              &= \lim_{\epsilon \to 0} \frac{1}{\epsilon} 
                    \left( e^{-\epsilon} \cdot \epsilon \cdot (f(x+1) - f(x)) +
                         \mathcal{O}(\epsilon^2) \right) \\
              & = f(x+1) - f(x)
    \end{aligned}$$
    So $Gf(x) = f(x + 1) - f(x)$ is the generator
        for the constant rate 1 Poisson process. 
     
    By the Markov property, 
    $$\begin{aligned}
        \lim_{\epsilon \to 0} \frac{1}{\epsilon}
            \E^x [ \E^s [f(N_{s+\epsilon}) - f(N_s) | N_s] ] = \E^x[Gf(N_s)]\,.
    \end{aligned}$$
    It follows from this and \eqref{condexp1} that
    $$\begin{aligned}\label{condexp2}
        \E^x[f(N_t)] = f(x) + \int_0^t \E^x [f(N_s + 1) - f(N_s) ] \dif s
    \end{aligned}$$
    
     
    Next we need the following notation.
    Define $(x)_m \defeq x(x-1) \cdots (x-m+1)$ and observe that 
    $$\begin{aligned}
        (x+1)_m - (x)_m
            &= (x+1) x (x-1) \cdots (x-m+2) - x (x-1) \cdots (x-m+1)\\
            &= ( (x+1) - (x-m+1) ) (x)_{m-1} \\
            &= m (x)_{m-1}\,.
    \end{aligned}$$
    Now by \eqref{condexp2},
    $$\begin{aligned}
        \E^0 [ (N_t)_m ] = 0 + \int_0^{t} \E^{0} [(N_s+1)_m - (N_s)_{m}] \dif s
                         = 0 + \int_0^{t} m \E^{0} [(N_s)_{m-1}] \dif s 
    \end{aligned}$$
        and $\E^0 [ (N_t)_0 ] = 1$. 
    
    Let 
    $$\begin{aligned}
        g_{m}(t) \defeq \int_0^{t} m \E^{0} [(N_s)_{m-1}] \dif s
    \end{aligned}$$ 
        and let $g_0(t) = 1$. 
    Then $g_{m}(t) = \int_0^{t} m g_{m-1}(s) \dif s$,
        so
    $$\begin{aligned}
        g_1(t) &= \E^0 [N_t]          =  \int_{0}^{t}  1   \dif s = t \\
        g_2(t) &= \E^0 [N_t(N_t - 1)] =  \int_{0}^{t}  2 s \dif s = t^2 \\
    \end{aligned}$$
        and by induction,
    $$\begin{aligned}
        \E^0 [(N_t)_m] = g_m(t) = t^m
    \end{aligned}$$
\end{example}


\begin{example}{What is the generator for Brownian motion?}
    Let $(B_t)_{t \geq 0}$ be a Brownian motion,
        ie. $B_{t+s} - B_t \sim \Normal(0, s)$. 
    Then
    $$\begin{aligned}
        \E^x[f(B_\epsilon)]
            &\approx \E^x[ f(x) + f'(x) (B_\epsilon - x) +
                \frac{1}{2}f''(x) (B_\epsilon - x)^2 + \cdots] \\
            &=       f(x) + f'(x)\cdot 0 + \frac{1}{2}f''(x) \cdot \epsilon +
                \mathcal{O}(\epsilon^{3/2})
    \end{aligned}$$
    So
    $$\begin{aligned}
        \lim_{\epsilon \to 0} \dfrac{1}{\epsilon}
            \E^{x}[ f(B_\epsilon) - f(x) ] = \dfrac{1}{2} f''(x)
    \end{aligned}$$
        and thus $Gf(x) = \dfrac{1}{2} f''(x)$ is the generator
        for standard Brownian motion.
\end{example}

\begin{example}{How can we make $N_t$ into Brownian motion?}
    We know that:
    \begin{enumerate}
        \item
        By the Central Limit Theorem, adding up independent noise,
            centering and scaling gets you the Gaussian distribution.
        
        \item
        $N_t$ is the number of points in a PPP on $[0, t]$,
            and so is a sum of a bunch of independent things.
        
        \item
        Brownian motion started at zero has $\E[B_t] = 0$, $\E[B_t^2] = t$.
    \end{enumerate}
    $N_t$ does not have enough noise in each interval to be Brownian.
    So consider $N_{M t}$.
    Since $\E[N_{M t}] = M t$, we can subtract $Mt$,
        and then divide by $\sqrt{M}$ to get the correct variance.
    Thus we define
    $$\begin{aligned}
        X_t^{(M)} = \dfrac{ N_{M t} - M t}{\sqrt{M}}
    \end{aligned}$$
    
    Let $G_M$ denote the generator of $X_t^{(M)}$.
    Then
    $$\begin{aligned}
        G_M f(x) = M \left(f \left(x + \frac{1}{\sqrt{M}}\right)
                            - f(x)\right) - \dfrac{1}{\sqrt{M}} f'(x)\,.
    \end{aligned}$$
    This is discussed in more detail on Day 15. 
    As $M \to \infty$, $G_M f(x) \to \dfrac{1}{2} f''(x)$. 
    Therefore, 
    $$\begin{aligned}
        (X^{(M)})_{t \geq 0} \xrightarrow[M \to \infty]{d} (B_t)_{t \geq 0}
    \end{aligned}$$
\end{example}


% ==============================================================================
%
\end{document}
%
% ==============================================================================
